\documentclass[11pt]{book}

\usepackage{amsmath}
\usepackage{amsfonts}
\usepackage{url}
\usepackage{amssymb,amsthm}
\usepackage[utf8]{inputenc}
%\usepackage{inconsolata}
\usepackage{sourcecodepro}

\usepackage{listings}

% Define Language
\lstdefinelanguage{futhark}
{
  % list of keywords
  morekeywords={
    do,
    else,
    fn,
    for,
    fun,
    if,
    in,
    let,
    loop,
    then,
    while,
    with,
  },
  sensitive=true, % keywords are not case-sensitive
  morecomment=[l]{--}, % l is for line comment
  morecomment=[s]{\{-}{-\}}, % s is for start and end delimiter
%  otherkeywords={>,<,=,<=,>=,!,*,/,-,+,|,&,||,&&,==,=>},
  morestring=[b]" % defines that strings are enclosed in double quotes
}

% Define Colors
\usepackage{color}
\definecolor{eclipseBlue}{RGB}{42,0.0,255}
\definecolor{eclipseGreen}{RGB}{63,127,95}
\definecolor{eclipsePurple}{RGB}{127,0,85}

% Set Language
\lstset{
  language={futhark},
  basicstyle=\ttfamily, % Global Code Style
  captionpos=b, % Position of the Caption (t for top, b for bottom)
  extendedchars=true, % Allows 256 instead of 128 ASCII characters
  tabsize=2, % number of spaces indented when discovering a tab
  columns=fixed, % make all characters equal width
  keepspaces=true, % does not ignore spaces to fit width, convert tabs to spaces
  showstringspaces=false, % lets spaces in strings appear as real spaces
  breaklines=true, % wrap lines if they don't fit
  frame=trbl, % draw a frame at the top, right, left and bottom of the listing
  frameround=tttt, % make the frame round at all four corners
  framesep=4pt, % quarter circle size of the round corners
  numbers=left, % show line numbers at the left
  numberstyle=\small\ttfamily, % style of the line numbers
  commentstyle=\slshape\bfseries\color{eclipseGreen}, % style of comments
  keywordstyle=\bfseries\color{eclipsePurple}, % style of keywords
  stringstyle=\color{eclipseBlue}, % style of strings
  emph=[1] {
    copy,
    filter,
    iota,
    map,
    partition,
    reduce,
    replicate,
    shape,
    scan,
    unzip,
    write,
    zip,
    zipWith,
  },
  emphstyle=[1]{\color{eclipseBlue}},
}


\usepackage{color}
\definecolor{eclipseBlue}{RGB}{42,0.0,255}
\newcommand{\soac}[1]{\texttt{\color{eclipseBlue}#1}}

\title{\bf Parallel Programming in Futhark}
\author{HIPERFIT \\ Department of Computer Science \\ University of Copenhagen (DIKU)}
\date{\today}

\begin{document}
\frontmatter
\maketitle
\chapter{Preface}

These notes ...

\tableofcontents
\mainmatter
\part{Parallel Functional Programming}
\chapter{Introduction}

\begin{enumerate}
\item Moores law, CPUs, GPUs, other parallel architectures
\item Concurrency vs parallelism
\item Task parallelism, data parallism, simd, mimd
\item Low-level languages vs high-level language approaches
\end{enumerate}

See \cite{finpar}.

\chapter{The Futhark Language}

\begin{lstlisting}
-- A least significant digit radix sort to test out `write`.
fun radix_sort_up(xs: [n]u32) : ([n]u32,[n]i32) =
  let is = iota(n) in
  loop (p:([n]u32,[n]i32) = (xs,is)) = for i < 32 do
    radix_sort_step_up(p,i)
  in p
\end{lstlisting}


\begin{enumerate}
\item Basic principles (sequentialising whole program compiler, fusion, optimisations for coallesced mem access)
\item core language
\item arrays and SOACs (forward ptr to reasoning about operator associativity, etc)
\item AoS to SoA
\item uniqueness types
\item modules
\end{enumerate}

\chapter{Algebraic Properties of SOACs}
\begin{enumerate}
\item general reasoning principles
\item assumptions
\item fusion rules
\item list homomorphism theorem
\item let the compiler do the fusion (how to reason)
\end{enumerate}

\chapter{Parallel Cost Models}
\begin{enumerate}
\item motivation
\item memory vs compute bound
\item nested parallism and flattening
\item work and depth
\item Futhark specifics and limitations
\end{enumerate}

\part{Parallel Algorithms}

\chapter{Parallel Algorithms}
In this chapter, we will present a number of parallel algorithms for
solving a number of problems. We will make use effective use of the
SOAC parallel operators, in particular, it turns out that the
\soac{scan} operator is critical for obtaining parallel algorithms. In
fact, we shall first develop the notion of a \emph{segmented scan}
operation, which, as we shall see, can be implemented using Futhark's \soac{scan}
operator, and which in its own right is essential to many of the later
algorithms.

\section{Segmented Scan}

\lstinputlisting[firstline=7]{src/sgm_scan.fut}

\begin{enumerate}
\item segmented scan
\item radix sort
\lstinputlisting[firstline=18]{src/radix_sort.fut}
\item pseudo random numbers and sobol
\item trees
\item graphs
\item longest streak
\item segmented replication
\item histograms
\item parenthesis matching
\end{enumerate}

\chapter{Bigger Applications}
\begin{enumerate}
\item monte carlo
\item learning with stochastic gradient descent
\item stencils
\item convolutions
\end{enumerate}

\chapter{Interoperability}
\begin{enumerate}
\item python and c
\item examples: mandelbrot, life, cam, nbody
\end{enumerate}

\bibliographystyle{plain}
\bibliography{bib}

\appendix

\part{Appendices}

\chapter{Tool References}
\begin{enumerate}
\item futhark-c, futhark-opencl
\item measuring runtimes, debugging
\end{enumerate}

\end{document}
